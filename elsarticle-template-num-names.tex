%% 
%% Copyright 2007-2020 Elsevier Ltd
%% 
%% This file is part of the 'Elsarticle Bundle'.
%% ---------------------------------------------
%% 
%% It may be distributed under the conditions of the LaTeX Project Public
%% License, either version 1.2 of this license or (at your option) any
%% later version.  The latest version of this license is in
%%    http://www.latex-project.org/lppl.txt
%% and version 1.2 or later is part of all distributions of LaTeX
%% version 1999/12/01 or later.
%% 
%% The list of all files belonging to the 'Elsarticle Bundle' is
%% given in the file `manifest.txt'.
%% 
%% Template article for Elsevier's document class `elsarticle'
%% with harvard style bibliographic references

\documentclass[preprint,12pt]{elsarticle}

%% Use the option review to obtain double line spacing
%% \documentclass[preprint,review,12pt]{elsarticle}

%% Use the options 1p,twocolumn; 3p; 3p,twocolumn; 5p; or 5p,twocolumn
%% for a journal layout:
%% \documentclass[final,1p,times]{elsarticle}
%% \documentclass[final,1p,times,twocolumn]{elsarticle}
%% \documentclass[final,3p,times]{elsarticle}
%% \documentclass[final,3p,times,twocolumn]{elsarticle}
%% \documentclass[final,5p,times]{elsarticle}
%% \documentclass[final,5p,times,twocolumn]{elsarticle}

%% For including figures, graphicx.sty has been loaded in
%% elsarticle.cls. If you prefer to use the old commands
%% please give \usepackage{epsfig}

%% The amssymb package provides various useful mathematical symbols
\usepackage{amssymb}
%% The amsthm package provides extended theorem environments
%% \usepackage{amsthm}

%% The lineno packages adds line numbers. Start line numbering with
%% \begin{linenumbers}, end it with \end{linenumbers}. Or switch it on
%% for the whole article with \linenumbers.
%% \usepackage{lineno}

\journal{Energy and AI}

\begin{document}

\begin{frontmatter}

%% Title, authors and addresses

%% use the tnoteref command within \title for footnotes;
%% use the tnotetext command for theassociated footnote;
%% use the fnref command within \author or \address for footnotes;
%% use the fntext command for theassociated footnote;
%% use the corref command within \author for corresponding author footnotes;
%% use the cortext command for theassociated footnote;
%% use the ead command for the email address,
%% and the form \ead[url] for the home page:
%% \title{Title\tnoteref{label1}}
%% \tnotetext[label1]{}
%% \author{Name\corref{cor1}\fnref{label2}}
%% \ead{email address}
%% \ead[url]{home page}
%% \fntext[label2]{}
%% \cortext[cor1]{}
%% \affiliation{organization={},
%%             addressline={},
%%             city={},
%%             postcode={},
%%             state={},
%%             country={}}
%% \fntext[label3]{}

\title{Coaxial multi-criteria optimization of a methane steam reforming reactor for effective hydrogen production and thermal management}

%% use optional labels to link authors explicitly to addresses:
%% \author[label1,label2]{}
%% \affiliation[label1]{organization={},
%%             addressline={},
%%             city={},
%%             postcode={},
%%             state={},
%%             country={}}
%%
%% \affiliation[label2]{organization={},
%%             addressline={},
%%             city={},
%%             postcode={},
%%             state={},
%%             country={}}

\author{Marcin Pajak}
\author{Grzegorz Brus}
\author{Janusz S. Szmyd}


\affiliation{organization={AGH University of Science and Technology},
            city={Krakow},
            country={Poland}}

\begin{abstract}
The advancement in environmental awareness is the recent driving factor of the energy industry development. The market sentiments dictate the commercialization of unconventional energy sources. Thus, generation via hydrogen conversion gains popularity. The presented research regards the enhancement of the steam reforming reaction, used for the production of hydrogen via the conversion of hydrocarbons. The reforming process characterizes by a strong endothermic nature. The rapid course of the reaction leads to the creation of temperature gradients of a considerable magnitude. The presented research strives to alleviate the negative consequences of the reaction character. An original strategy by the name of macro-patterning is suggested as a remedy. The presented research proposes an updated concept, predicting the introduction of coaxial segments to the catalytic insert. The segments may consist of catalytic material or metallic foam applied for local suppression of the reaction. The morphology of specific segments may be altered independently, to allow for additional control of the reforming reaction. The objective of the research is to define the optimal segment composition. The optimization process is based on an in-house procedure implementing a genetic algorithm. The acquired results appear to validate the macro-pattering concept. A significant unification of the temperature field is obtained, with a simultaneous increase in hydrogen productivity.
\end{abstract}

%%Graphical abstract
\begin{graphicalabstract}
%\includegraphics{grabs}
\end{graphicalabstract}

%%Research highlights
\begin{highlights}
\item Introduction of a novel approach to the macro-patterning concept.
\item Sensitivity analysis conducted for the evolutionary algorithm parameters.
\item Enhancement of thermal conditions via a modification of the catalyst insert. 
\item Increase in hydrogen productivity. 
\end{highlights}

\begin{keyword}
hydrogen \sep evolutionary algorithms \sep reforming \sep design optimization
%% keywords here, in the form: keyword \sep keyword

%% PACS codes here, in the form: \PACS code \sep code

%% MSC codes here, in the form: \MSC code \sep code
%% or \MSC[2008] code \sep code (2000 is the default)

\end{keyword}

\end{frontmatter}

%% \linenumbers

%% main text
\section{}
\label{}

%% The Appendices part is started with the command \appendix;
%% appendix sections are then done as normal sections
%% \appendix

%% \section{}
%% \label{}

%% For citations use: 
%%       \citet{<label>} ==> Jones et al. [21]
%%       \citep{<label>} ==> [21]
%%

%% If you have bibdatabase file and want bibtex to generate the
%% bibitems, please use
%%
%%  \bibliographystyle{elsarticle-num-names} 
%%  \bibliography{<your bibdatabase>}

%% else use the following coding to input the bibitems directly in the
%% TeX file.

\begin{thebibliography}{00}

%% \bibitem[Author(year)]{label}
%% Text of bibliographic item

\bibitem[ ()]{}

\end{thebibliography}
\end{document}

\endinput
%%
%% End of file `elsarticle-template-num-names.tex'.
